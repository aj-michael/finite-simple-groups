\documentclass{beamer}
\usepackage{graphicx}
\usepackage{mathtools}
\usepackage{amssymb}
\usetheme{CambridgeUS}


\title[]
{Classification of Finite Simple Groups}
\author
{Adam Michael}
\institute[Rose-Hulman]
{
  Department of Mathematics\\
  Rose-Hulman Institute of Technology
}
\date[Febuarary 22, 2016] % (optional)
{MA376 Abstract Algebra, 2016}
\subject{Topology}

\begin{document}
\frame{\titlepage}

\begin{frame}
    \frametitle{Everything we want to talk about}
    \begin{itemize}
    	\item V - E + F = 2
        \item $\chi(P) = V - E + F$
        \item $\chi(X)=\sum_{\alpha}(-1)^{dim(\sigma_\alpha)}$
        \item $2\pi\chi(X) = \int_M \kappa dA$
        \item $\chi(M) = \sum_{i} index_{x_i}(v)$
    \end{itemize}
\end{frame}

\begin{frame}
	\frametitle{Euler's Polyhedron Formula}
	\begin{itemize}
		\item V - E + F = 2
		\item Discovered independently by Descartes (1630) and Euler (1750)
		\item Euler's proof was only valid for convex solids
		\item Prior to this point, very little math was done that did not involve measurement
		\item $\chi = V - E + F$
	\end{itemize}
\end{frame}

\begin{frame}
	\frametitle{Euler's Formula}
\end{frame}

\begin{frame}
    \frametitle{Counterexample to Classical Euler's Formula}
    In 1813 Lhuilier published an important work. He noticed that Euler's formula was wrong for solids with holes in them.
\end{frame}

\begin{frame}
	\frametitle{Counterexample to Classical Euler's Formula}
	\begin{itemize}[<+->]
		\item This suggests the existence of a more general version of the Euler's Formula.
		\item If a solid has g holes, Lhuilier showed that
		\[
		    V - E + F = 2 - 2g
		\]
		\item The quantity $\mathcal{X}(g) = 2 - 2g$ is called Euler Characteristics.
		\item This was the first known result on a topological invariant.
	\end{itemize}
\end{frame}

\begin{frame}
    \frametitle{Euler's Theorem}
    In 1847 Karl Georg Christian von Staut corrected Euler's Theorem.
    \vspace*{\baselineskip}
    
    \begin{theorem}[von Staut]
        Let $P$ be a polyhedron which satisfies
        \begin{itemize}{}
            \item Any two vertices of $P$ can be connected by a chain of edges
            \item Any loop on $P$ which is made up of straight line segments separates $P$ into two pieces
        \end{itemize}
        Then $V - E + F = 2$
    \end{theorem}
\end{frame}

\begin{frame}
    \frametitle{An application!}
    Suppose you are a designer of soccer balls. Every ball is to be made of $H$ white hexagons and $P$ black pentagons, stitched together with the edges lining up. Suppose that every vertex must join exactly three shapes. Can we find any restrictions on $H$ or $P$?
\end{frame}

\begin{frame}
    \frametitle{Lots of definitions}
    \begin{definition}
        A $k$-simplex is the $k$-dimensional convex hull of a set of $k+1$ points in $\mathbb{R}^k$
        \begin{itemize}
            \item 0-simplex = point
            \item 1-simplex = line
            \item 2-simplex = triangle
            \item 3-simplex = tetrahedron
            \item 4-simplex = ``5 cell'', 4-dimensional object bounded by 5 tetrahedrons
        \end{itemize}
    \end{definition}
    \begin{definition}
        An $m$-face of an $n$-simplex where $m < n$ is the convex hull of $m+1$ points chosen from the $n+1$ points defining the $n$-simplex.
    \end{definition}
\end{frame}

\begin{frame}
    \frametitle{Lots of definitions}
    \begin{columns}
        \column{0.5 \textwidth}
            \begin{definition}
                A simplicial complex $\kappa$ is a topology that is a set of simplices satisfying
                \begin{itemize}
                    \item Any face of a simplex in $\kappa$ is also in $\kappa$
                    \item For any two simplices $\sigma_1, \sigma_2 \in \kappa$, either $\sigma_1 \cap \sigma_2 = \emptyset$ or $\sigma_1 \cap \sigma_2$ is a face of both $\sigma_1$ and $\sigma_2$
                \end{itemize}
            \end{definition}
        \column{0.5 \textwidth}
    \end{columns}

\end{frame}
\begin{frame}
    \frametitle{Even more definitions}
    \begin{definition}
        A triangulation of a topological space $X$ is a simplicial complex $\kappa$ and a homeomorphism $h: \kappa \rightarrow X$
    \end{definition}
\end{frame}

\begin{frame}
    \frametitle{Back to the Euler Characteristic}
    \begin{definition}
        Given a space $X$ with a triangulation $(\kappa, h)$ where $\kappa$ has finite faces $\sigma_\alpha$, the Euler characteristic of $X$ is defined as $$\chi(X)=\sum_{\alpha}(-1)^{dim(\sigma_\alpha)}$$
    \end{definition}
    \begin{itemize}
        \item This quantity is independent of the triangularization of $X$
        \item If $X$ is a polyhedron, this definition is equivalent to $\chi(X) = V - E + F$
    \end{itemize}
\end{frame}

%% Here comes the CW-Complex
\begin{frame}
	\frametitle{Here comes the CW-Complex}
	\begin{definition}[CW-Complex]
		We construct the space with the following procedures. 
		\begin{enumerate}
			\item Start with a discrete set $X_0$ , whose points are regarded as 0-cells.
			\item Inductively, form the n-skeleton $X^n$ from $X^{n - 1}$ by attaching n-cells $e^n_\alpha$ via maps $\phi_\alpha: S^{n-1} \to X^{n-1}$. This means that $X^n$ is the quotient space of the disjoint union $X^{n-1} \sqcup_\alpha D^n_\alpha$ of $X^{n-1}$ with a collection of n-disks under the identifications $x ~ \phi_\alpha(x)$ for $x \in \partial D^n_\alpha$.
		\end{enumerate}
	\end{definition}
	Forget the formal definitions. Let's have a look at what it means.
\end{frame}

\begin{frame}
	\frametitle{Everything is CW-Complex}
	\begin{theorem}[CW-Approximation Theorem]
		If $X$ is any space, then there is a CW-complex $Y$ and a map $f:Y \to X$ inducing isomorphisms on all homotopy, homology, and cohomology groups.
	\end{theorem}
	Everything is just a conglomeration of CW-complexes!
\end{frame}

%% Gauss-Bonnet slides starting here
\begin{frame}
	\frametitle{Gauss-Bonnet Theorem}
	\begin{theorem}[Classical Gauss-Bonnet Theorem]
		For $M$ a compact smooth oriented surface in $\mathbb{R}^3$, the integral of Gauss Curvature with respect to area on M equals
		\[
			\int_M \kappa dA = 2\pi \mathcal{X}(M)
		\]
	\end{theorem}
	\pause
	\begin{theorem}[Gauss-Bonnet Theorem in Simplex]
		\[
			\int_M d\kappa = \int_{M^(0)} d\kappa + \int_{M^(1)} d\kappa + \int_{M^(2)} d\kappa = 2\pi \mathcal{X}(M)
		\]
		where $M^{(n)}$ denotes the n-cell of $M$.
	\end{theorem}
	\pause
	What does these all mean?
\end{frame}

%% Descartes Total Angular Defect
\begin{frame}
	\frametitle{Curvature in 0-Cell}
	In 0-cell vertex, the curvature $\kappa$ is simply the angular defect on that vertex.
	\begin{definition}[Angular Defect]
		The angular defect of a vertex is the difference between the sum of face angles $A_i$ at a polyhedron vertex of a polyhedron and $2 \pi$,
		\[
			\delta = 2 \pi - \sum_i A_i
		\]
	\end{definition}
	\begin{theorem}[Descartes Total Angular Defect]
		The total angular defect $\Delta$ is the sum of the angular defects over all polyhedron vertices of a polyhedron
		\[
			\Delta = \sum_i \delta_i = 4 \pi
		\]
	\end{theorem}
\end{frame}

\begin{frame}{Gauss-Bonnet}
    \frametitle{A dip into Differential Geometry}
    \begin{columns}
    	\column{0.5 \textwidth}
    		What is the sum of the external angles of a polygon?
    	\column{0.5 \textwidth}
    		\pause
    		And what is the sum of the angles of the normal-bounded sectors?
    \end{columns}
\end{frame}

\begin{frame}{Gauss-Bonnet}
    \frametitle{Curvature in 1-Cell}
    \begin{definition}[Curvature]
    	The \textbf{curvature} of a plane curve at a point is
    	\[
    		\frac{d\theta}{ds} = \lim_{\Delta x \to 0} \Big( \frac{\Delta \theta}{\Delta s} \Big)
    	\]
    	where $\Delta \theta$ is the angle between the tangents at points separated by an arc of the curve of length $\Delta s$.
    \end{definition}
    \vspace{\baselineskip}
    \begin{columns}
    	\column{0.5 \textwidth}
    	\pause
		\column{0.5 \textwidth}
		\pause
			It follows from this definition that $\int_C \kappa ds = 2 \pi$ for a simple closed curve C
    \end{columns}
\end{frame}

\begin{frame}{Gauss-Bonnet}
    \frametitle{A dip into Differential Geometry}
    \begin{definition}
    	The \textbf{exterior solid angle} at each vertex P is the (area of the) sector of a unit ball bounded by planes normal to the edges at P
    \end{definition}
    \pause
    The second method can be extended from polygon to polyhedra, by replacing the normal bounded sector with exterior solid angle.
    \pause
\end{frame}

\begin{frame}
	\frametitle{Curvature in 2-cell}
	\begin{theorem}
		Let $S$ be a smooth closed convex surface. Then
		\[
			\iint_S \kappa_1 \kappa_2 dA = 4 \pi
		\]
		where $\kappa_1 \kappa_2$ is the Gaussian Curvature.
	\end{theorem}
	\begin{columns}
    	\column{0.5 \textwidth}
    	\pause
    	    The surface can be imagined as a polyhedra with infinitely many partitions.
		\column{0.5 \textwidth}
		\pause
    \end{columns}
\end{frame}

%% Poincare-Hopf Theorem
\begin{frame}
    \frametitle{From topology to analysis}
    \begin{theorem}[Poincare-Hopf]
        Let $v$ be a vector field on $M$, where $M$ is without boundary, which also only has isolated zeros. Then the sum of the indices of the zeros of the vector field is equal to the Euler characteristic of the manifold.
    \end{theorem}
    \begin{itemize}
        \item First by Poincare, 1885, expanded by Hopf, 1926.
        \item Relates a purely topological concept (Euler characteristic) to a purely analytical concept (index of a vector field).\cite{dummit}
    \end{itemize}
\end{frame}

\begin{frame}
	\frametitle{References}
	\bibliographystyle{ieeetr}
	\bibliography{bibliography}
\end{frame}
\end{document}