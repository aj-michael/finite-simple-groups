\documentclass[a4paper]{article}

\usepackage[utf8]{inputenc}
\usepackage{amsmath}
\usepackage{amsfonts}
\usepackage{amsthm}
\usepackage{graphicx}

\title{Classification of Finite Simple Groups}

\author{Adam Michael}

\date{May 5, 2016}

\begin{document}
\maketitle

\newtheorem{theorem}{Theorem}
\newtheorem{lemma}{Lemma}
\newtheorem{definition}{Definition}

\begin{theorem}
All of the following groups are simple and every finite simple group is isomorphic to one of the following:
\begin{enumerate}
	\item
	A cyclic group $Z_p$ of prime order $p$
	\item
	An alternating group $A_n$ for $n \ge 5$
	\item
	A classical group for $q$ a prime-power:
	\begin{itemize}
		\item
		Linear: $PSL_n(q)$, $n \ge 2$, except $PSL_2(2)$ and $PSL_2(3)$
		\item
		Unitary: $PSU_n(q)$, $n \ge 3$, except $PSU_3(2)$
		\item
		Symplectic: $PSp_{2n}(q)$, $n \ge 2$, except $PSp_4(2)$
		\item
		Orthogonal: $P\Omega_{2n+1}(q)$, $n \ge 3$, $q$ odd\\
		$P\Omega^+_{2n}(q)$, $n \ge 4$\\
		$P\Omega^-_{2n}(q)$, $n \ge 4$
	\end{itemize}
	\item
	An exceptional group of Lie Type for $q$ a prime-power:
	\begin{itemize}
		\item $G_2(q)$, $q \ge 3$
		\item $F_4(q)$
		\item $E_6(q)$
		\item ${}^2E_6(q)$
		\item ${}^3D_4(q)$
		\item $E_7(q)$
		\item $E_8(q)$
		\item ${}^2B_2(2^{2n+1})$, $n \ge 1$
		\item ${}^2G_2(3^{2n+1})$, $n \ge 1$
		\item ${}^2F_4(2^{2n+1})$, $n \ge 1$
	\end{itemize}
	or the Tits group ${}^2F_4(2)'$.
	\item
	One of the 26 sporadic simple groups:
	\begin{itemize}
		\item
		The Mathieu groups $M_{11}, M_{12}, M_{22}, M_{23}, M_{24}$
		\item
		The Leech lattice groups $Co_1, Co_2, Co_3, McL, HS, Suz, J_2$
		\item
		The Fischer groups $Fi_{22}, Fi_{23}, Fi'_{24}$
		\item
		The monstrous groups $\mathbb{M}, \mathbb{B}, Th, HB, He$
		\item
		The pariahs $J_1, J_3, J_4, O'N, Ly, Ru$.
	\end{itemize}
\end{enumerate}
\end{theorem}

\begin{theorem}For prime $p$, $\mathbb{Z}_p$ is simple.
\end{theorem}
\begin{proof}
Since $\mathbb{Z}_p$ has $p$ elements, by LaGrange's Theorem all subgroups of $\mathbb{Z}_p$ have order dividing $p$. Since $p$ is prime, the only subgroups of $\mathbb{Z}_p$ are $\mathbb{Z}_p$ and $\{0\}$. Therefore $\mathbb{Z}_p$ has no nontrivial proper subgroups, so it is simple.
\end{proof}

\begin{theorem}
If G is a simple nontrivial abelian group, then $G \cong \mathbb{Z}_p$ from some prime $p$.
\end{theorem}
\begin{proof}
Consider $x \in G$, $x \neq e$. Then $\{ e \} \neq \langle x \rangle \le G$. $G$ is abelian so $\langle x \rangle \triangleleft G$. Then since $G$ is simple, $\langle x \rangle = G$. If $|\langle x \rangle| = \infty$, then $\langle x \rangle < \langle x^2 \rangle \leq G$, which contradicts $\langle x \rangle = G$. Thus If $|\langle x \rangle| = p$ for some finite $p$. Therefore $G \cong \mathbb{Z}_p$. Suppose $p$ is composite, that is $p = m n$, $m > 1, n > 1$. Then $Z_n \leq Z_p \cong G$, so G has a nontrivial proper subgroup. But since $G$ is abelian, this contradicts the supposition that $G$ is simple. Therefore $p$ must be prime.
\end{proof}

\begin{theorem}
\label{generatethm}
For $n \ge 3$, $A_n$ is exactly the set of products of 3-cycles on $n$ letters.
\end{theorem}
\begin{proof}~
\begin{enumerate}
\item[$\Leftarrow$]
Let $\sigma = \sigma_1 \sigma_2 ... \sigma_k$ be a product of 3-cycles. Since each $\sigma_i$ is even, $\sigma$ is even. So $\sigma \in A_n$.
\item[$\Rightarrow$]
Let $\sigma \in A_n$. Write $\sigma$ as a product of transpositions. Since $\sigma$ is even, $\sigma=(a_1\ b_1)(c_1\ d_1)...(a_k\ b_k)(c_k\ d_k)$. Consider $(a_i\ b_i)(c_i\ d_i)$. Since $(a_i\ b_i)$ and $(c_i\ d_i)$ are transpositions, it must be the case that $a_i \ne b_i$ and $c_i \ne d_i$. If $a_i, b_i, c_i, d_i$ are all distinct, then $(a_i\ b_i)(c_i\ d_i) = (a_i\ b_i\ c_i)(b_i\ c_i\ d_i)$. If $a_i, b_i, c_i, d_i$ contain one duplicate letter, without loss of generality (or by relabeling) suppose $a_i = c_i$. Then $(a_i\ b_i)(a_i\ d_i) = (a_i\ d_i\ b_i)$. If $a_i, b_i, c_i, d_i$ contain two duplicate letters, then $(a_i\ b_i)(a_i\ b_i) = ()$. Thus each pair of transpositions in $\sigma$ can be replaced by a product of 3-cycles. Therefore $\sigma$ is a product of 3-cycles.
\end{enumerate}
\end{proof}

\begin{definition}
For a group $G$, if $a, b \in G$ then $b$ is conjugate to $a$ if there exists $g \in G$ such that $g a g^{-1} = b$.
\end{definition}

\begin{theorem}
\label{conjugatetheorem}
For every pair of 3-cycles $c_1, c_2 \in A_n$, $c_2$ is conjugate to $c_1$.
\end{theorem}
\begin{proof}
Let $c_1 = (a\ b\ c)$. Proceed by cases on the number of letters shared by $c_1$ and $c_2$.
\begin{description}
   \item[Case 1:]$c_2$ shares no letters with $c_1$.
   Then $c_2 = (d\ e\ f)$ and $a, b, c, d, e, f$ are all distinct. Let $s = (d\ e)(e\ f)(b\ d)(a\ e) \in A_n$. Then $s c_1 s^{-1} = (d\ e)(e\ f)(b\ d)(a\ e)(a\ b\ c)(a\ e)(b\ d)(e\ f)(d\ e) = (d\ e\ f) = c_2$.
   \item[Case 2:]$c_2$ shares one letter with $c_1$.
   Relabel so that $c_2 = (a\ d\ e)$. Let $s = (b\ d)(c\ e) \in A_n$. Then $s c_1 s^{-1} = (b\ d)(c\ e)(a\ b\ c)(c\ e)(b\ d) = (a\ d\ e) = c_2$.
   \item[Case 3:]$c_2$ shares two letters with $c_1$.
   Relabel so that $c_2 = (a\ b\ d)$. Let $s = (c\ d)(d\ e) \in A_n$. Then $s c_1 s^{-1} = (c\ d)(d\ e)(a\ b\ c)(d\ e)(c\ d) = (a\ b\ d) = c_2$.
   \item[Case 4:]$c_2$ shares three letters with $c_1$.
   Then $c_2 = c_1$. Let $s = () \in A_n$. Then $() c_1 ()^{-1} = c_1 = c_2$.
\end{description}
\end{proof}

\begin{lemma}
\label{threecyclelemma}
If $N \triangleleft A_5$ and $N \ne \{()\}$ then $N$ contains a 3-cycle.
\end{lemma}
\begin{proof}
Since $N \ne \{()\}$, there exists $\sigma \in N$, $\sigma \ne ()$. Since $\sigma \ne ()$, $\sigma$ is even and $\sigma$ acts on 5 letters, there are three possibilities for $\sigma$ written as a product of disjoint cycles.
\begin{description}
    \item[Case 1:]
    $\sigma = (a\ b\ c)$. Then $\sigma$ is a 3-cycle in $N$.
    \item[Case 2:]
    $\sigma = (a\ b\ c\ d\ e)$. Let $\alpha = (a\ b)(c\ d) \in A_n$. Since $N \triangleleft A_n$, $\sigma \alpha \sigma \alpha^{-1} \in N$. $\sigma \alpha \sigma \alpha^{-1} = (a\ b\ c\ d\ e)(a\ b)(c\ d)(a\ b\ c\ d\ e)(a\ b)(c\ d) = (a\ e\ c)$. So $(a\ e\ c)$ is a 3-cycle in $N$.
    \item[Case 3:]
    $\sigma = (a\ b)(c\ d)$. Let $\alpha = (a\ b\ e) \in A_n$. Since $N \triangleleft A_n$, $\sigma \alpha \sigma \alpha^{-1} \in N$. $\sigma \alpha \sigma \alpha^{-1} = (a\ b)(c\ d)(a\ b\ e)(a\ b)(c\ d)(a\ e\ b) = (a\ b\ e)$. So $(a\ b\ e)$ is a 3-cycle in $N$.
\end{description}
\end{proof}

\begin{theorem}
\label{a5theorem}
$A_5$ is simple.
\end{theorem}
\begin{proof}
Suppose $A_5$ has a nontrivial normal subgroup $N \triangleleft A_5$. By Lemma \ref{threecyclelemma}, $N$ contains a 3-cycle $c_1$. Let $c_2$ be a 3-cycle in $A_n$. Then by Theorem \ref{conjugatetheorem}, there exists $s \in A_n$ such that $s c_1 s^{-1} = c_2$. Then since $N$ is normal $c_2 \in N$. Therefore $N$ contains all 3-cycles in $A_n$. Then since $N$ is closed, by Theorem \ref{generatethm} we have $N = A_n$. Therefore $A_n$ is simple.
\end{proof}

\begin{lemma}
\label{fixedpointlemma}
For $n \ge 5$, if $\sigma \in A_n$, $\sigma \ne ()$, then there exists $\sigma' \in A_n$, $\sigma' \ne \sigma$ conjugate to $\sigma$ such that for some $i \in \{1, ..., n\}$, $\sigma(i) = \sigma'(i)$.
\end{lemma}
\begin{proof}
Let $\sigma \in A_n$, $\sigma \ne ()$. Write $\sigma$ as the product of disjoint cycles and let $r$ be the length of the longest disjoint cycle in $\sigma$. Since $\sigma \ne ()$, $r \ge 2$. If $r = 2$, then $\sigma$ is the product of a non-zero even number of disjoint transpositions. Proceed by cases.
\begin{description}
	\item[Case 1:]$r = 2$ and $\sigma$ is the product of two disjoint transpositions.
	Then $\sigma = (a\ b)(c\ d)$. Let $\alpha = (a\ c\ b)$ and $\sigma' = \alpha \sigma \alpha^{-1} = (a\ c\ b)(a\ b)(c\ d)(a\ b\ c) = (a\ c)(b\ d)$. Since $n \ge 5$, there exists $e \notin \{a, b, c, d\}$. So $\sigma \ne \sigma'$ and $\sigma(e) = \sigma'(e) = e$.
	\item[Case 2:]$r = 2$ and $\sigma$ is the product of more than two disjoint transpositions.
	Then write $\sigma = (a\ b)(c\ d)(e\ f)\pi$ where $\pi$ is a permutation that fixes $a, b, c, d, e, f$. Let $\alpha = (a\ b)(c\ e)$ and $\sigma' = \alpha \sigma \alpha^{-1}$. Then $\sigma' = (a\ b)(c\ e)(a\ b)(c\ d)(e\ f)\pi(a\ b)(c\ e) = (a\ b)(c\ f)(d\ e)\pi$. So $\sigma \ne \sigma'$ and $\sigma(a) = \sigma'(a) = b$.
	\item[Case 3:]$r > 2$.
	Then with relabeling, $\sigma = (1\ 2\ 3\ ...\ r)\pi$ where $\pi$ is a permutation that fixes $1$ through $r$. Let $\alpha = (3\ 4\ 5)$ and $\sigma' = \alpha \sigma \alpha^{-1} = (3\ 4\ 5)(1\ 2\ 3\ ...\ r)\pi(3\ 5\ 4)$. $\sigma(2) = 3$ and $\sigma'(2) = 4$ so $\sigma \ne \sigma'$ and $\sigma(1) = \sigma'(1) = 2$.
\end{description}
% TODO: Write this proof
\end{proof}

\begin{theorem}
For $n \ge 5$, $A_n$ is simple.
\end{theorem}
\begin{proof}
By induction on $n$. \\
Base case: $n = 5$. Proved as Theorem \ref{a5theorem}. \\
Induction hypothesis: Suppose $A_{k-1}$ is simple. \\
Now consider $A_k$. For $i \in \{1, ..., k\}$ let $H_i = \{\sigma \in A_k\ |\ \sigma(i) = i\}$. Then $H_i \cong A_{k-1}$. By the induction hypothesis, $H_i$ is simple.
% TODO: Finish this proof
\end{proof}

\begin{thebibliography}{9}
\bibitem{nano3}
  K. Grove-Rasmussen og Jesper Nygård,
  \emph{Kvantefænomener i Nanosystemer}.
  Niels Bohr Institute \& Nano-Science Center, Københavns Universitet

\end{thebibliography}
\end{document}
