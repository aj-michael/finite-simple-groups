\documentclass[a4paper]{article}

\usepackage[utf8]{inputenc}
\usepackage{amsmath}
\usepackage{amsfonts}
\usepackage{amssymb}
\usepackage{amsthm}
\usepackage{graphicx}

\title{Classification of Finite Simple Groups}

\author{Adam Michael}

\date{May 5, 2016}

\begin{document}
\maketitle

\newtheorem{theorem}{Theorem}
\newtheorem{lemma}{Lemma}
\newtheorem{definition}{Definition}
\newtheorem{example}{Example}

In Abstract Algebra, we give special attention to normal subgroups. Normal subgroups have several interesting properties, including equality of left and right cosets and the ability to construct quotient groups. A natural extension of the study of normal subgroups is the problem of finding the normal subgroups of a given group. For a group $G$, both the trivial group and $G$ are normal subgroups of $G$ and if $G$ is abelian then every subgroup is normal. But in general, it is not at all obvious if $G$ has any other normal subgroups. This gives rise to the study of simple groups and their classification.

\begin{definition}
A nontrivial group $G$ is simple if its only normal subgroups are the trivial group and $G$.
\end{definition}

I will focus specifically on finite simple groups. The study of finite simple groups is a rich and far-reaching subject. I will give a brief history of the subject, state the most significant major result and provide proofs of a few smaller results.

The study of finite simple groups began with Galois around 1830 when he was studying the solutions by radicals to polynomial equations \cite{gallian}. Galois proved that the alternating groups on at least five letters are simple. In 1870, Camille Jordan constructed the projective special linear  groups $PSL_n(q)$ and proved they are simple for prime-power $q$. In 1872, Ludwig Sylow proved the Sylow Theorems which were among the first results used to classify finite simple groups \cite{wilson}. The Sylow Theorems are given below without proof. An example demonstrates how they are used to study finite simple groups.

\begin{definition}
If $G$ is a group, $H_1 \le G$, $H_2 \le G$, $g \in G$ and $g H_1 g^{-1} = H_2$ then $H_1$ and $H_2$ are called conjugate.
\end{definition}

\begin{theorem}[Sylow Theorems \cite{wilson}]
If $G$ is a finite group of order $p^k n$, $p$ prime and $p \nmid n$, then
\begin{enumerate}
	\item $G$ has subgroups of order $p^k$, called Sylow $p$-subgroups
	\item The Sylow $p$-subgroups are all conjugate
	\item The number $s_p$ of Sylow $p$-subgroups is such that $s_p \mid m$ and $s_p \equiv 1 \mod p$	
\end{enumerate}
\end{theorem}

\begin{theorem}
\label{unprovedtheorem}
If $G$ is a finite group and $S$ is a Sylow $p$-subgroup, $S$ is normal in $G$ if and only if it is the unique Sylow $p$-subgroup.
\end{theorem}
\begin{proof}~
\begin{enumerate}
\item[$\Rightarrow$]
Suppose $S$ and $S'$ are normal Sylow $p$-subgroups. By the second Sylow Theorem, there exists $g \in G$ such that $g S g^{-1} = S'$. $S$ is normal so $g S g^{-1} = S$. So $S = S'$.
\item[$\Leftarrow$]
Suppose $S$ is the unique Sylow $p$-subgroup. Let $g \in G$. Note that $|g S g^{-1}| = |S|$ because $S$ is a group. Consider $g a g^{-1}, g b g^{-1} \in g S g^{-1}$. $g a g^{-1} g b g^{-1} = g (a b) g^{-1} \in g S g^{-1}$. So by the finite subgroup test, $g S g^{-1}$ is a subgroup of order $|S|$. So $g S g^{-1} = S$. Therefore $S$ is normal.
\end{enumerate}
\end{proof}

\begin{example}
Let $G$ be a group of order $84 = 2^2 \cdot 3 \cdot 7$. Then by the third Sylow Theorem, $s_7 \mid 12$. So $s_7 \in \{1, 2, 3, 4, 6, 12\}$. Also, $s_7 \equiv 1 \mod 7$, so $s_7 = 1$. Then by Theorem \ref{unprovedtheorem}, the unique Sylow 7-subgroup is normal in $G$. Therefore there are no simple groups of order 84.
\end{example}

As seen in the example above, the Sylow Theorems are useful for proving that all groups of specific orders have a nontrivial proper normal subgroup. In 1963, Walter Feit and John Thompson proved another result for ruling out orders of simple groups. It deals with solvable groups, which for the sake of brevity I will not define.

\begin{theorem}[Feit-Thompson Theorem]
Every finite group of odd order is solvable. As corollary, every non-abelian finite simple group has even order.
\end{theorem}

As a result of the Feit-Thompson Theorem, group theorists began to believe that all finite abelian groups had been found. This sparked a large-scale effort to prove a complete classification theorem for finite simple groups. Over the next two decades, about twenty new finite simple groups were found. In the 1980s, it was generally believed that all finite simple groups had been found and that a proof could be compiled. The resulting Classification Theorem for Finite Simple Group spans tens of thousands of pages across publications by over 100 authors.

\begin{theorem}[Classification Theorem for Finite Simple Groups (CTFSG) \cite{wilson}]
The following groups are simple and every finite simple group is isomorphic to one of them:
\begin{enumerate}
	\item
	A cyclic group $Z_p$ of prime order $p$
	\item
	An alternating group $A_n$ for $n \ge 5$
	\item
	A classical group for $q$ a prime-power:
	\begin{itemize}
		\item
		Linear: $PSL_n(q)$, $n \ge 2$, except $PSL_2(2)$ and $PSL_2(3)$
		\item
		Unitary: $PSU_n(q)$, $n \ge 3$, except $PSU_3(2)$
		\item
		Symplectic: $PSp_{2n}(q)$, $n \ge 2$, except $PSp_4(2)$
		\item
		Orthogonal: $P\Omega_{2n+1}(q)$, $n \ge 3$, $q$ odd\\
		$P\Omega^+_{2n}(q)$, $n \ge 4$\\
		$P\Omega^-_{2n}(q)$, $n \ge 4$
	\end{itemize}
	\item
	An exceptional group of Lie Type for $q$ a prime-power:
	\begin{itemize}
		\item $G_2(q)$, $q \ge 3$
		\item $F_4(q)$
		\item $E_6(q)$
		\item ${}^2E_6(q)$
		\item ${}^3D_4(q)$
		\item $E_7(q)$
		\item $E_8(q)$
		\item ${}^2B_2(2^{2n+1})$, $n \ge 1$
		\item ${}^2G_2(3^{2n+1})$, $n \ge 1$
		\item ${}^2F_4(2^{2n+1})$, $n \ge 1$
	\end{itemize}
	or the Tits group ${}^2F_4(2)'$.
	\item
	One of 26 sporadic simple groups:
	\begin{itemize}
		\item
		The Mathieu groups $M_{11}, M_{12}, M_{22}, M_{23}, M_{24}$
		\item
		The Leech lattice groups $Co_1, Co_2, Co_3, McL, HS, Suz, J_2$
		\item
		The Fischer groups $Fi_{22}, Fi_{23}, Fi'_{24}$
		\item
		The monstrous groups $\mathbb{M}, \mathbb{B}, Th, HB, He$
		\item
		The pariahs $J_1, J_3, J_4, O'N, Ly, Ru$.
	\end{itemize}
\end{enumerate}
\end{theorem}

The scope of CTFSG is surprising to say the least. Simple groups are similar to prime numbers in the sense that groups can be factored into quotient groups in much the same way that integers are factored into their prime factorization. (Note that there is not necessarily a unique way to factor a group into quotient groups.) However there is no known analogous classification theorem for prime numbers. Imminent completion of the proof of CTFSG was announced around 1980. Gorenstein, Lyons and Solon have attempted to consolidate the entire proof into one publication. Six of the eleven proposed "volumes" of the proof have been published with the most recent being published in 2005 \cite{number6}. Solomon has estimated that the total proof will consist of approximately 5,000 pages. Due to the sheer size of the proof, I will only discuss a small part of it. I will start with the classification of finite abelian simple groups.



\begin{theorem}For prime $p$, $\mathbb{Z}_p$ is simple.
\end{theorem}
\begin{proof}
Since $\mathbb{Z}_p$ has $p$ elements, by LaGrange's Theorem all subgroups of $\mathbb{Z}_p$ have order dividing $p$. Since $p$ is prime, the only subgroups of $\mathbb{Z}_p$ are $\mathbb{Z}_p$ and $\{0\}$. Therefore $\mathbb{Z}_p$ has no nontrivial proper subgroups, so it is simple.
\end{proof}

\begin{theorem}
If G is a simple nontrivial abelian group, then $G \cong \mathbb{Z}_p$ from some prime $p$.
\end{theorem}
\begin{proof}
Consider $x \in G$, $x \neq e$. Then $\{ e \} \neq \langle x \rangle \le G$. $G$ is abelian so $\langle x \rangle \triangleleft G$. Then since $G$ is simple, $\langle x \rangle = G$. If $|\langle x \rangle| = \infty$, then $\langle x \rangle < \langle x^2 \rangle \leq G$, which contradicts $\langle x \rangle = G$. Thus If $|\langle x \rangle| = p$ for some finite $p$. Therefore $G \cong \mathbb{Z}_p$. Suppose $p$ is composite, that is $p = m n$, $m > 1, n > 1$. Then $Z_n \leq Z_p \cong G$, so G has a nontrivial proper subgroup. But since $G$ is abelian, this contradicts the supposition that $G$ is simple. Therefore $p$ must be prime.
\end{proof}

Thus we have classified all finite abelian simple groups as exactly the groups $\mathbb{Z}_p$ for prime $p$. The classification for abelian groups is far simpler than for nonabelian groups. This is not surprising, as the Fundamental Theorem of Finite Abelian Groups states that all finite abelian groups are isomorphic to the direct product of prime-power cyclic groups. The direct product of two prime-power cyclic groups contains a subgroup isomorphic to one of the prime-power cyclic groups, so it is not simple. I will now show that the alternating groups on more than four letters are simple. The proof for $A_n$ is more complicated than that for $\mathbb{Z}_p$, so it is broken into multiple theorems.

\begin{theorem}
\label{generatethm}
For $n \ge 3$, $A_n$ is exactly the set of products of 3-cycles on $n$ letters.
\end{theorem}
\begin{proof}~
\begin{enumerate}
\item[$\Leftarrow$]
Let $\sigma = \sigma_1 \sigma_2 ... \sigma_k$ be a product of 3-cycles. Since each $\sigma_i$ is even, $\sigma$ is even. So $\sigma \in A_n$.
\item[$\Rightarrow$]
Let $\sigma \in A_n$. Write $\sigma$ as a product of transpositions. Since $\sigma$ is even, $\sigma=(a_1\ b_1)(c_1\ d_1)...(a_k\ b_k)(c_k\ d_k)$. Consider $(a_i\ b_i)(c_i\ d_i)$. Since $(a_i\ b_i)$ and $(c_i\ d_i)$ are transpositions, it must be the case that $a_i \ne b_i$ and $c_i \ne d_i$. If $a_i, b_i, c_i, d_i$ are all distinct, then $(a_i\ b_i)(c_i\ d_i) = (a_i\ b_i\ c_i)(b_i\ c_i\ d_i)$. If $a_i, b_i, c_i, d_i$ contain one duplicate letter, without loss of generality (or by relabeling) suppose $a_i = c_i$. Then $(a_i\ b_i)(a_i\ d_i) = (a_i\ d_i\ b_i)$. If $a_i, b_i, c_i, d_i$ contain two duplicate letters, then $(a_i\ b_i)(a_i\ b_i) = ()$. Thus each pair of transpositions in $\sigma$ can be replaced by a product of 3-cycles. Therefore $\sigma$ is a product of 3-cycles.
\end{enumerate}
\end{proof}

\begin{definition}
For a group $G$, if $a, b \in G$ then $b$ is conjugate to $a$ if there exists $g \in G$ such that $g a g^{-1} = b$.
\end{definition}

\begin{theorem}
\label{conjugatetheorem}
For every pair of 3-cycles $c_1, c_2 \in A_n$, $c_2$ is conjugate to $c_1$.
\end{theorem}
\begin{proof}
Let $c_1 = (a\ b\ c)$. Proceed by cases on the number of letters shared by $c_1$ and $c_2$.
\begin{description}
   \item[Case 1:]$c_2$ shares no letters with $c_1$.
   Then $c_2 = (d\ e\ f)$ and $a, b, c, d, e, f$ are all distinct. Let $s = (d\ e)(e\ f)(b\ d)(a\ e) \in A_n$. Then $s c_1 s^{-1} = (d\ e)(e\ f)(b\ d)(a\ e)(a\ b\ c)(a\ e)(b\ d)(e\ f)(d\ e) = (d\ e\ f) = c_2$.
   \item[Case 2:]$c_2$ shares one letter with $c_1$.
   Relabel so that $c_2 = (a\ d\ e)$. Let $s = (b\ d)(c\ e) \in A_n$. Then $s c_1 s^{-1} = (b\ d)(c\ e)(a\ b\ c)(c\ e)(b\ d) = (a\ d\ e) = c_2$.
   \item[Case 3:]$c_2$ shares two letters with $c_1$.
   Relabel so that $c_2 = (a\ b\ d)$. Let $s = (c\ d)(d\ e) \in A_n$. Then $s c_1 s^{-1} = (c\ d)(d\ e)(a\ b\ c)(d\ e)(c\ d) = (a\ b\ d) = c_2$.
   \item[Case 4:]$c_2$ shares three letters with $c_1$.
   Then $c_2 = c_1$. Let $s = () \in A_n$. Then $() c_1 ()^{-1} = c_1 = c_2$.
\end{description}
\end{proof}

\begin{lemma}
\label{threecyclelemma}
If $N \triangleleft A_5$ and $N \ne \{()\}$ then $N$ contains a 3-cycle.
\end{lemma}
\begin{proof}
Since $N \ne \{()\}$, there exists $\sigma \in N$, $\sigma \ne ()$. Since $\sigma \ne ()$, $\sigma$ is even and $\sigma$ acts on 5 letters, there are three possibilities for $\sigma$ written as a product of disjoint cycles.
\begin{description}
    \item[Case 1:]
    $\sigma = (a\ b\ c)$. Then $\sigma$ is a 3-cycle in $N$.
    \item[Case 2:]
    $\sigma = (a\ b\ c\ d\ e)$. Let $\alpha = (a\ b)(c\ d) \in A_n$. Since $N \triangleleft A_n$, $\sigma \alpha \sigma \alpha^{-1} \in N$. $\sigma \alpha \sigma \alpha^{-1} = (a\ b\ c\ d\ e)(a\ b)(c\ d)(a\ b\ c\ d\ e)(a\ b)(c\ d) = (a\ e\ c)$. So $(a\ e\ c)$ is a 3-cycle in $N$.
    \item[Case 3:]
    $\sigma = (a\ b)(c\ d)$. Let $\alpha = (a\ b\ e) \in A_n$. Since $N \triangleleft A_n$, $\sigma \alpha \sigma \alpha^{-1} \in N$. $\sigma \alpha \sigma \alpha^{-1} = (a\ b)(c\ d)(a\ b\ e)(a\ b)(c\ d)(a\ e\ b) = (a\ b\ e)$. So $(a\ b\ e)$ is a 3-cycle in $N$.
\end{description}
\end{proof}

\begin{theorem}
\label{a5theorem}
$A_5$ is simple.
\end{theorem}
\begin{proof}
Suppose $A_5$ has a nontrivial normal subgroup $N \triangleleft A_5$. By Lemma \ref{threecyclelemma}, $N$ contains a 3-cycle $c_1$. Let $c_2$ be a 3-cycle in $A_n$. Then by Theorem \ref{conjugatetheorem}, there exists $s \in A_n$ such that $s c_1 s^{-1} = c_2$. Then since $N$ is normal $c_2 \in N$. Therefore $N$ contains all 3-cycles in $A_n$. Then since $N$ is closed, by Theorem \ref{generatethm} we have $N = A_n$. Therefore $A_n$ is simple.
\end{proof}

\begin{lemma}
\label{fixedpointlemma}
For $n \ge 5$, if $\sigma \in A_n$, $\sigma \ne ()$, then there exists $\sigma' \in A_n$, $\sigma' \ne \sigma$ conjugate to $\sigma$ such that for some $i \in \{1, ..., n\}$, $\sigma(i) = \sigma'(i)$.
\end{lemma}
\begin{proof}
Let $\sigma \in A_n$, $\sigma \ne ()$. Write $\sigma$ as the product of disjoint cycles and let $r$ be the length of the longest disjoint cycle in $\sigma$. Since $\sigma \ne ()$, $r \ge 2$. If $r = 2$, then $\sigma$ is the product of a non-zero even number of disjoint transpositions. Proceed by cases.
\begin{description}
	\item[Case 1:]$r = 2$ and $\sigma$ is the product of two disjoint transpositions.
	Then $\sigma = (a\ b)(c\ d)$. Let $\alpha = (a\ c\ b)$ and $\sigma' = \alpha \sigma \alpha^{-1} = (a\ c\ b)(a\ b)(c\ d)(a\ b\ c) = (a\ c)(b\ d)$. Since $n \ge 5$, there exists $e \notin \{a, b, c, d\}$. So $\sigma \ne \sigma'$ and $\sigma(e) = \sigma'(e) = e$.
	\item[Case 2:]$r = 2$ and $\sigma$ is the product of more than two disjoint transpositions.
	Then write $\sigma = (a\ b)(c\ d)(e\ f)\pi$ where $\pi$ is a permutation that fixes $a, b, c, d, e, f$. Let $\alpha = (a\ b)(c\ e)$ and $\sigma' = \alpha \sigma \alpha^{-1}$. Then $\sigma' = (a\ b)(c\ e)(a\ b)(c\ d)(e\ f)\pi(a\ b)(c\ e) = (a\ b)(c\ f)(d\ e)\pi$. So $\sigma \ne \sigma'$ and $\sigma(a) = \sigma'(a) = b$.
	\item[Case 3:]$r > 2$.
	Then with relabeling, $\sigma = (1\ 2\ 3\ ...\ r)\pi$ where $\pi$ is a permutation that fixes $1$ through $r$. Let $\alpha = (3\ 4\ 5)$ and $\sigma' = \alpha \sigma \alpha^{-1} = (3\ 4\ 5)(1\ 2\ 3\ ...\ r)\pi(3\ 5\ 4)$. $\sigma(2) = 3$ and $\sigma'(2) = 4$ so $\sigma \ne \sigma'$ and $\sigma(1) = \sigma'(1) = 2$.
\end{description}
\end{proof}

\begin{theorem}
For $n \ge 5$, $A_n$ is simple.
\end{theorem}
\begin{proof}
By induction on $n$. \\
Base case: $n = 5$. Proved as Theorem \ref{a5theorem}. \\
Induction hypothesis: Suppose $A_{k-1}$ is simple for $k \ge 6$. \\
Now consider $A_k$. Suppose $A_k$ has a nontrivial normal subgroup $N \triangleleft A_k$ and there exists $\sigma \in N$, $\sigma \ne ()$. By Lemma \ref{fixedpointlemma}, there exists $\sigma' \in A_k$ conjugate to $\sigma$ and $i \in \{1, ..., k\}$ such that $\sigma \ne \sigma'$ and $\sigma(i) = \sigma'(i)$. This implies that $\sigma^{-1} \sigma' \ne ()$. Since $N$ is normal, $\sigma' \in N$ and $N$ is closed so $\sigma^{-1} \sigma' \in N$. By multiplication by inverse on the left of $\sigma(i) = \sigma'(i)$, $(\sigma^{-1} \sigma')(i) = i$. Now let $H_i = \{s \in A_k\ |\ s(i) = i\} \le A_k$. Then $H_i \cong A_{k-1}$. By the induction hypothesis, $H_i$ is simple. So $\sigma^{-1} \sigma' \in H_i \cap N$. Since $H_i$ and $N$ are both normal subgroups of $A_k$, $H_i \cap N \triangleleft G$. $\sigma^{-1} \sigma' \in H_i \cap N$ and $H_i$ is simple so $H_i \cap N = H_i$. Thus $H_i \subseteq N$. Since $H_i \cong A_{k-1}$ and $k \ge 6$, $A_{k-1}$ contains a 3-cycle so $H$ contains a 3-cycle. Therefore $N$ contains a 3-cycle, $c_1 \in N$. Let $c_2 \in A_k$ be another 3-cycle. Then by Theorem \ref{conjugatetheorem} there exists $s \in A_k$ such that $s c_1 s^{-1} = c_2$. Then since $N$ is normal, $c_2 \in N$. So $N$ contains all 3-cycles in $A_k$. Then by Theorem \ref{generatethm} we have $N=A_k$. Therefore $A_k$ is simple.
\end{proof}

% TODO: Insert conclusion

\begin{thebibliography}{9}
\bibitem{gallian}  
  Gallian, Joseph A. \emph{Contemporary Abstract Algebra}. Boston, MA: Brooks/Cole Cengage Learning, 2013. Print.
\bibitem{number6}
  Gorenstein, Daniel, Richard Lyons, and Ronald Solomon. \emph{The Classification of the Finite Simple Groups Number 6}. Providence, RI: American Mathematical Society, 2005. Print.
\bibitem{wilson}
  Wilson, Robert. \emph{The Finite Simple Groups}. London New York: Springer, 2009. Print.

\end{thebibliography}
\end{document}
